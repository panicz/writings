\documentclass{beamer}

\newenvironment{tightcenter}{%
  \setlength\topsep{0pt}
  \setlength\parskip{0pt}
  \begin{center}
}{%
  \end{center}
}

\mode<presentation>
{
  \usetheme{Copenhagen}
  \usecolortheme[RGB={173,222,25}]{structure}
  \setbeamertemplate{items}[circle]
  \setbeamercovered{transparent}
}

\usepackage[polish]{babel}

\usepackage[utf8]{inputenc}
\usepackage{times}
\usepackage[T1]{fontenc}
\title{\textbf{Programowanie funkcyjne w PHP}}

\author{Maciek Godek}

\institute{\{ \textbf{fido} : \emph{labs} \}\newline \newline
\tiny{\href{mailto:godek.maciek@gmail.com}{\textbf{godek.maciek@gmail.com}}}}

\date{\textbf{PHP3City Meetup\#4}, 06.12.2013}

\begin{document}

\begin{frame}
  \titlepage
\end{frame}

\section{Wprowadzenie}

\subsection{Czym jest programowanie funkcyjne?}

\begin{frame}{Czym jest programowanie funkcyjne?}

  \textbf{Programowanie funkcyjne (FP) polega na:}
  \pause
  \begin{itemize}
  \item
    unikaniu skutków ubocznych,
    \pause
  \item
    stosowaniu funkcji wyższego rzędu.
  \end{itemize}
\end{frame}

\subsection{Dlaczego zainteresować się FP?}

\begin{frame}{Sformułowanie problemu}

  \textbf{Jaki problem chcemy rozwiązać?}\pause \newline
  - Uczynić programowanie jak maksymalnie efektywnym, \pause czyli:
  \begin{itemize}
    \item
      musieć jak najmniej pamiętać, \pause oraz
    \item
      móc jak najwięcej wnioskować.
  \end{itemize}
\end{frame}

\begin{frame}{Sformułowanie problemu}
  Innymi słowy: \pause \textbf{uczynić program jak najbardziej czytelnym}
\end{frame}

\begin{frame}{Sposoby na usprawnienie pracy z kodem}
  
  Pomysły:\pause
  \begin{itemize}
    \item
      korzystanie z Integrated Development Environment\pause
    \item
      używanie statycznej kontroli typów\pause
    \item
      stosowanie konwencji w nazewnictwie obiektów\pause
    \item
      ogólniej: \textbf{przestrzeganie konwencji w sposobie użycja języka}
  \end{itemize}
  
\end{frame}

\begin{frame}{Konwencje w sposobie użycia języka}
  Kilka popularnych konwencji:\pause
  \begin{itemize}
    \item
      notacja węgierska\pause
    \item
      metodologia OOP (hermetyzacja, enkapsulacja, polimorfizm, 
      dziedziczenie, ...)\pause
    \item
      \textbf{programowanie funkcyjne}
  \end{itemize}
\end{frame}

\section{Przykłady}

\subsection{Spójniki}

\begin{frame}{Zagadka: co robi poniższy kod?}
\pause
\texttt{
\begin{tabbing}
\textbf{function} złoncz(\emph{\$słowa}) \{ \\
\hspace{8 mm} \= \emph{\$wynik} = ''; \\
\> \emph{\$n} = count(\emph{\$słowa}); \\
\> \textbf{if}(\emph{\$n} < 1) \{ \textbf{return} \emph{\$wynik}; \} \\
\> \emph{\$bieżące} = reset(\emph{\$słowa}); \\
\> \textbf{for}(\emph{\$i} = 0; \emph{\$i} < \emph{\$n}; \emph{\$i}++) \{ \\
\> \hspace{8 mm} \= \emph{\$wynik} .= \emph{\$bieżące}; \\
\> \> \textbf{if}(\emph{\$i} < \emph{\$n}-2) \{ \emph{\$wynik} .= ', '; \} \\
\> \> \textbf{elseif}(\emph{\$i} == \emph{\$n}-2) \{ \emph{\$wynik} .= ' i '; \} \\
\> \> \emph{\$bieżące} = next(\emph{\$słowa}); \\
\> \} \\
\> \textbf{return} \emph{\$wynik}; \\
\}
\end{tabbing}
}
\end{frame}

\begin{frame}{Rozwiązanie: łączenie słów naturalnymi spójnikami}
\textbf{Rozwiązanie:} złącza słowa spójnikami, w taki sposób,
w jaki łączymy słowa w mowie potocznej, tzn. sklejamy
wszystkie słowa przecinkami, za wyjątkiem dwóch ostatnich,
które sklejamy spójnikiem 'i'.
\end{frame}

\begin{frame}{Przykład użycia}
\texttt{
\begin{tabbing}
złączenie([]) = ''; \\
złączenie(['Ola']) = 'Ola'; \\
złączenie(['Ola', 'Ala']) = 'Ola i Ala'; \\
złączenie(['Ola', 'Ala', 'Ula']) = 'Ola, Ala i Ula'; \\
złączenie(['Ola', 'Ala', 'Ula', 'Ela']) \\
\hspace{8 mm}= 'Ola, Ala, Ula i Ela';
\end{tabbing}
}
\end{frame}

\begin{frame}{Abstrakcja!}
\texttt{
\begin{tabbing}
złączenie([]) = ''; \\
złączenie([\emph{\$a}]) = \emph{\$a}; \\
złączenie([\emph{\$a}, \emph{\$b}]) = \emph{\$a} . ' i ' .\emph{\$b}; \\
złączenie([\emph{\$a}, \emph{\$b}, \emph{\$c}, ...]) \\
\hspace{8 mm}= \emph{\$a} . ', ' . złączenie([\emph{\$b}, \emph{\$c}, ...]);
 \\
\end{tabbing}
}
\end{frame}

\begin{frame}{W stronę PHP}
\texttt{
\begin{tabbing}
\textbf{function} złączone(\emph{\$słowa}) \{ \emph{// marzeń} \\
\hspace{8 mm} \= \textbf{match}(\emph{\$słowa}) \{ \\
\> \textbf{case} []: \\
\> \hspace{8 mm} \= \textbf{return} ''; \\
\> \textbf{case} [\emph{\$a}]: \\
\> \> \textbf{return} \emph{\$a}; \\
\> \textbf{case} [\emph{\$a}, \emph{\$b}]: \\
\> \> \textbf{return} \emph{\$a} . ' i ' . \emph{\$b}; \\
\> \textbf{case} [\emph{\$a}, \emph{\$b}, \emph{\$c}, ...]: \\
\> \> \textbf{return} \emph{\$a} . ', ' \\
\> \> \hspace{8 mm} . złączone([\emph{\$b}, \emph{\$c}, ...]); \\
\> \} \\
\}
\end{tabbing}
}
\end{frame}

\begin{frame}{Faktyczna implementacja w PHP}
\texttt{
\begin{tabbing}
\textbf{function} złączone(\emph{\$słowa}) \{ \\
\hspace{8 mm} \= \textbf{switch}(count(\emph{\$słowa})) \{ \\
\> \textbf{case} 0: \\
\> \hspace{8 mm} \= \textbf{return} ''; \\
\> \textbf{case} 1: \\
\> \> \textbf{return} array\_pop(\emph{\$a}); \\
\> \textbf{default}: \\
\> \> \emph{\$ostatnie} = array\_pop(\emph{\$słowa}); \\
\> \> \textbf{return} implode(', ', \emph{\$słowa}) \\
\> \> \hspace{8 mm} . ' i ' . \emph{\$ostatnie}; \\
\> \} \\
\} \\
\end{tabbing}
}
\end{frame}


\subsection{Sortowanie tablicy}

\begin{frame}{Kolejna zagadka!}

\Large{Jak posortować tablicę \uncover<2->{stringów}\newline
w PHP \uncover<2->{według ich długości}?}
\end{frame}

\begin{frame}{Pierwsza próba:}
\texttt{
\begin{tabbing}
usort(\emph{\$tablica}, \textbf{function}(\emph{\$a}, \emph{\$b}) \{ \\ 
\hspace{8 mm}\textbf{return} length(\emph{\$a}) - length(\emph{\$b}); \\
\});
\end{tabbing}
}
\pause
Pytanie: \textbf{czy powyższe wywołanie posortuje tablicę rosnąco,
czy malejąco?}
\end{frame}

\begin{frame}{Druga próba:}
\textbf{\texttt{usort(\emph{\$tablica}, ascending('length'))}}
\pause , \newline \newline
gdzie
\texttt{
\begin{tabbing}
\textbf{function} ascending(\emph{\$property}) \{ \\
\hspace{8 mm} \= \textbf{return} \textbf{function}(\emph{\$a}, \emph{\$b})
 \textbf{use}(\emph{\$property}) \{ \\
\> \hspace{8 mm} \= \textbf{return} \emph{\$property}(\emph{\$a})
 - \emph{\$property}(\emph{\$b}); \\
\> \}; \\
\}
\end{tabbing}
}
\end{frame}

\begin{frame}{Sortowanie malejąco}
\texttt{
\begin{tabbing}
\textbf{function} descending (\emph{\$prop}) \{ \\
\hspace{8 mm} \= \textbf{return} \textbf{function}(\emph{\$a}, \emph{\$b}) 
\textbf{use}(\emph{\$prop}) \{ \\
\> \hspace{8 mm} \= \textbf{return} -call\_user\_func(ascending(\emph{\$prop}),\\
\> \> \hspace{56 mm} \emph{\$a}, \emph{\$b}); \\
\> \}; \\
\}
\end{tabbing}
}
\pause
\begin{tabbing}
jeżeli \texttt{is\_callable(\$f)}, \\
\hspace{8 mm} \= to zapis \texttt{call\_user\_func(\$f, \$argumenty ...)} \\
\> jest równoważny zapisowi \texttt{\$f(\$argumenty ...)}
\end{tabbing}
\end{frame}


\subsection{Inne funkcje wyższego rzędu}
\begin{frame}{Popularne funkcje wyższego rzędu}
  \begin{itemize}
  \item
    \texttt{call\_user\_func\_array(\$function, \$array)}
    \pause
  \item
    \texttt{array\_map(\$callback, \$a1, \$a2, ...)}
    \pause
  \item
    \texttt{array\_filter(\$array, \$callback)}
    \pause
  \item
    \texttt{array\_reduce(\$callback, \$array [, \$init])}
  \end{itemize}
\end{frame}


\begin{frame}{\textbf{\texttt{call\_user\_func\_array}} -- synonimy 1}
  Zastosowanie \texttt{call\_user\_func\_array} -- tworzenie synonimów:
\pause
\texttt{
\begin{tabbing}
\textbf{function} apply() \{ \\
\hspace{8 mm} \= \textbf{return} call\_user\_func\_array( \\
\> \hspace{8 mm} \= 'call\_user\_func\_array', \\
\> \> func\_get\_args() \\
\> ); \\
\} \\
 \\
 \\
 \\
 \\
 \\
\end{tabbing}
}
\end{frame}


\begin{frame}{\textbf{\texttt{call\_user\_func\_array}} -- synonimy 2}
  Zastosowanie \texttt{call\_user\_func\_array} -- tworzenie synonimów:
\pause
\texttt{ \begin{tabbing} \textbf{function} map() \{ \\
\hspace{8 mm} \textbf{return} apply('array\_map', func\_get\_args()); \\
\} \end{tabbing} } \pause \texttt{ \begin{tabbing}
\textbf{function} filter() \{ \\
\hspace{8 mm} \textbf{return} apply('array\_filter', func\_get\_args()); \\
\} \end{tabbing} } \pause \texttt{ \begin{tabbing}
\textbf{function} reduce() \{ \\
\hspace{8 mm} \textbf{return} apply('array\_reduce', func\_get\_args()); \\
\}
\end{tabbing}
}
\end{frame}

\begin{frame}{\textbf{\texttt{call\_user\_func\_array}} -- synonimy 3}
  Zastosowanie \texttt{call\_user\_func\_array} do tworzenia synonimów -- 
\textbf{problemy}: \pause
  \begin{itemize}
  \item
    \alert{brak argumentów formalnych} -- utrudnia współpracę z IDE
    \pause
  \item
    \alert{brak obsługi referencji}
    \pause
    \texttt{ \begin{tabbing} \textbf{function} \alert{pop}() \{ \\
        \hspace{8 mm} \textbf{return} apply('array\_pop', func\_get\_args()); \\
        \} \textbf{\alert{// źle!!!}} \end{tabbing} } 
    \pause
    \item
      \alert{generuje narzuty wywołań} (kod działa wolniej)
      \pause
      \item
        a co z obsługą argumentów domyślnych?
  \end{itemize}
\end{frame}


\begin{frame}{\textbf{\texttt{call\_user\_func\_array}} -- synonimy 4}
  Zagadka: \textbf{jak stworzyć synonim dla} \texttt{func\_get\_args}?
\end{frame}


\begin{frame}{\textbf{\texttt{array\_map}} -- przykład}
\texttt{map}: odwzorowanie zbioru funkcją. \pause Na przykład, chcąc 
usunąć białe znaki ze stringów będących elementami tablicy przy pomocy
funkcji trim, możemy napisać: \newline \newline
\texttt{\$ogolone\_napisy = map('trim', \$napisy)}
\end{frame}

\begin{frame}{\textbf{\texttt{array\_map}} -- zagadka}
Zagadka: \textbf{Chcemy wiedzieć, jaka jest największa długość
stringu w danej tablicy (stringów).}
\end{frame}

\begin{frame}{\textbf{\texttt{array\_map}} -- rozwiązanie}
Rozwiązanie: \texttt{max(map('length', \emph{\$tablica}))}
\newline \newline \pause
\textbf{Uwaga!} To jest po prostu zdanie ,,największa spośród długości
elementów w \texttt{\$tablicy}'' wyrażone w PHP!
\end{frame}


\begin{frame}{\textbf{\texttt{array\_map}} -- ciekawostka}
\textbf{Ciekawostka:} \newline 
\texttt{map('f', map('g', \$a)) = map('f' $\circ$ 'g', \$a)}, gdzie
symbol $\circ$ oznacza złożenie funkcji, tj. $f(g(x))$
\newline \newline \pause
\textbf{Zadanie:} napisz funkcję 'compose', która pobiera jako
argumenty funkcje jednoargumentowe i zwraca funkcję będącą ich 
złożeniem
\end{frame}

\begin{frame}{\textbf{\texttt{array\_filter}} -- przykład}
\texttt{filter}: odsiewanie elementów spełniających dane kryterium\pause, 
np. chcemy znaleźć wszystkie liczby z danej tablicy, mieszczące się
w przedziale (2, 5) \pause
\texttt{
\begin{tabbing}
\$liczby\_2\_5 = filter(\$różne\_liczby, \textbf{function}(\$x) \{ \\
\hspace{8mm} \textbf{return} (2 < \$x) \&\& (\$x < 5); \\
\});
\end{tabbing}
}
\end{frame}

\begin{frame}{\textbf{\texttt{array\_reduce}} -- wzmianka}
\texttt{reduce}: funkcyjna bestia \newline \pause
\textbf{zastosowanie}: uogólnianie dwuargumentowych działań łącznych na
dowolną ilość argumentów
\end{frame}

\subsection{Łączenie OOP i FP}
\begin{frame}{Składnia \textbf{\texttt{\$\_}}}
\textbf{Problem:} mamy tablicę obiektów. Chcielibyśmy mieć tablicę rzutów 
tych obiektów na ich pewne własności, które możemy otrzymać przez wywołanie
odpowiedniej metody. \newline \pause
Klasyczne rozwiązanie:
\texttt{
\begin{tabbing}
\$rzuty = []; \\
foreach(\$obiekty as \$obiekt) \{ \\
\hspace{8 mm} \$rzuty[] = \$obiekt->pobierzWłasność(); \\
\}
\end{tabbing}
} \pause 
Funkcyjne rozwiązanie:
\texttt{
\begin{tabbing}
\uncover<4->{global \$\_; // drobne oszustwo} \\
\$rzuty = map(\$\_->pobierzWłasność(), \$obiekty);
\end{tabbing}
}
\end{frame}

\begin{frame}{Składnia \textbf{\texttt{\$\_}}}
\textbf{Inny problem:} sortowanie tablicy tablic względem danej kolumny:\newline
\pause
\texttt{usort(\$array, ascending(\$\_['column']));}
\end{frame}

\section*{Podsumowanie}

\begin{frame}{Podsumowanie}
  \pause
  Zalety FP:
  \pause
  \begin{itemize}
  \item
    łatwiejsza analiza kodu
    \pause
  \item
    dobra kompozycjonalność (nie trzeba się przejmować kopiowaniem i referencjami)
    \pause
  \item
    bardzo naturalne testowanie
    \pause
  \item
    uproszczona paralelizacja
  \end{itemize}
  \pause
  Wady FP:
  \pause
  \begin{itemize}
  \item
    nie wszędzie daje się stosować
    \pause
  \item
    brak wsparcia ze strony narzędzi
    \pause
  \item
    kod może działać niewydajnie
  \end{itemize}
\end{frame}

\begin{frame}{Lektury}
  Wpiszcie sobie w Google: \newline
  \emph{John Carmack Functional Programming}\newline
  \newline \pause
  
  Książki: \newline
  ,,Struktura i Interpretacja Programów Komputerowych'', H.~Abelson, G.~Sussman
\end{frame}


\begin{frame}{Dziękuję!}
\begin{tightcenter}
\large{\emph{https://joind.in/10258}}\newline
\tiny{\textbf{godek.maciek@gmail.com}}
\end{tightcenter}
\end{frame}


\end{document}
