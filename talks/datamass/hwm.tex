\documentclass{beamer}

\newenvironment{tightcenter}{%
  \setlength\topsep{0pt}
  \setlength\parskip{0pt}
  \begin{center}
}{%
  \end{center}
}

\mode<presentation>
{
  \usetheme{Copenhagen}
  %%\usecolortheme[RGB={173,222,25}]{structure}
  \usecolortheme[RGB={66,134,244}]{structure}
  \setbeamertemplate{items}[circle]
  \setbeamercovered{transparent}
}

\usepackage[polish]{babel}
\usepackage{chessfss}
\PassOptionsToPackage{hyphens}{url}
\usepackage{hyperref}
\usepackage{qtree}
\usepackage{mathtools}
\usepackage{dirtytalk}
\usepackage{epigraph}
\usepackage{textgreek}
\usepackage[utf8]{inputenc}
\usepackage{times}
\usepackage[T1]{fontenc}
\usepackage{tikz}
\usepackage{csquotes}
\usepackage{amsmath}
\usepackage{fancyvrb}
\usepackage{ulem}
\usepackage{adjustbox}

\newcommand{\prompt}{\phantom{}>\phantom{}>\phantom{}>\ }

\newenvironment{Snippet}{\Verbatim[samepage=true]}{\endVerbatim}

\title{\textbf{The Hassle with Monads}}

\author{Panicz Maciej Godek}

\institute{
  \tiny{\href{mailto:godek.maciek@gmail.com}{\textbf{godek.maciek@gmail.com}}} \\
  \normalsize{\url{https://github.com/panicz/writings/tree/master/talks/datamass}}
}

\date{\textbf{datamass.io summit}, 28.09.2018}

\begin{document}

\begin{frame}
  \titlepage
\end{frame}

\begin{frame}{Before we begin - questions to the audience:}
  \begin{itemize} \pause
  \item do you program? \pause
  \item do you not program?
  \end{itemize}
\end{frame}

\begin{frame}{Which languages do you know?}
  \begin{itemize} \pause
  \item Java/C\# \pause
  \item Python \pause
  \item JavaScript \pause
  \item C/C++ \pause
  \item Scala \pause
  \item Erlang/Elixir \pause
  \item OCaml/F\# \pause
  \item Haskell \pause
  \item some other?
  \end{itemize}
\end{frame}

\begin{frame}{On to the topic}
  \begin{itemize} \pause
  \item do you know functional programming? \pause
  \item have you heard about monads? \pause
  \item do you understand monads? \pause
  \item did you try to understand monads and failed?
  \end{itemize}
\end{frame}


\begin{frame}{Why learn monads?}

  \begin{displayquote}
    Yes, monads seem to be a form of AspectOrientedProgramming,
    since they serve to isolate a generalized computational
    strategy from the specifics of an algorithm. For example,
    in HaskellLanguage you can write a graph-searching
    procedure that can either do a depth-first search
    and return the first result, or do a breadth-first
    search and return a list of results, merely by running
    it in a different monad.
  \end{displayquote}

  \begin{flushright}
    {\footnotesize \url{http://wiki.c2.com/?AspectOrientedProgramming}}
  \end{flushright}
  
\end{frame}

\begin{frame}{Why learn monads?}
  \includegraphics[width=\textwidth]{hello-haskell.jpg}
\end{frame}


\begin{frame}{Why (lazy) functional programming?}
  \includegraphics[width=\textwidth]{recursive-centaur.png}
\end{frame}

\begin{frame}{Why (lazy) functional programming?}
  \texttt{numbersFrom n = n:numbersFrom(n+1)}\\ \pause
  \texttt{numbersFrom 0 = [0,1,2,3,4,5,6,7,8,9,10,...]}\\ \pause
  \ \\
  \texttt{sieve (h:t) = h:(sieve [x|x<-t,x`mod`h/=0])} \\ \pause
  \texttt{primes = sieve (numbersFrom 2)} \\ \pause
  \texttt{primes = [2,3,5,7,11,13,17,19,23,29,31,37,...]} \\ \pause
  \ \\ \
  \\ Jerzy Karczmarczuk, \textit{Generating Power of Lazy Semantics} \\ \pause  John Hughes, \textit{Why Functional Programming Matters}
\end{frame}

\begin{frame}{The essence of laziness}
  \texttt{square x = x * x} \\ \pause
  \ \\
  Applicative order (evaluate arguments before expansion): \\ \pause
  \texttt{square (2*3) \pause = square 6 \pause =$_{def}$ 6 * 6 \pause = 36} \\ \pause
  \ \\
  Normal order (evaluate arguments after expansion): \\ \pause
  \texttt{square (2*3) \pause =$_{def}$ (2*3) * (2*3) \pause = 6 * 6 \pause = 36}
\end{frame}

\begin{frame}{Lambda the Ultimate}
  \texttt{square x = x * x} \\ \pause
  \texttt{square = λ x -> x * x} \\ \pause
  \texttt{square = function(x) \{ return x * x; \} } \\ \pause
  \ \\
  \texttt{\textbf{let} name = value \textbf{in} expression} \\ \pause
  \texttt{(λ name -> expression) value}
\end{frame}

\begin{frame}{The problem with I/O}
  \texttt{1*3 + 2*0} \\ \pause
  \ \\
  \texttt{readNumber()*3 + 2*readNumber()} \\ \pause
  \texttt{\phantom{}< 1} \\
  \texttt{\phantom{}< 0} \\
\end{frame}

\begin{frame}{Attempted solution}
  \texttt{ \\ \ \\
    let \ a\ \ \ \ \  = readNumber(\ \ ) in \\ \pause
    \ \ let \ b\ \ \ \ \ = readNumber(\ \ ) in \\ \pause
    \ \ \ \ \ a*2 + 3*b
  } \\ \ \\ \ \\ \ 
\end{frame}

\begin{frame}{Actual solution}
  \texttt{ \\ \ \\
    let (a,w1) = readNumber(w0) in \\
    \ \ let (b,w2) = readNumber(w1) in \\
    \ \ \ \ \ a*2 + 3*b
  } \\ \ \\ \ \\ \ 
\end{frame}

\begin{frame}{Better (composable) solution}
  \texttt{ \\ \ \\
    let (a,w1) = readNumber(w0) in \\
    \ \ let (b,w2) = readNumber(w1) in \\
    \ \ \ \ (a*2 + 3*b, w2)
  } \\ \ \\ \ \\ \ 
\end{frame}

\begin{frame}{A function}
  \texttt{ \\
    myOperation w0 = \\
    let (a,w1) = readNumber(w0) in \\
    \ \ let (b,w2) = readNumber(w1) in \\
    \ \ \ \ (a*2 + 3*b, w2)
  } \\ \ \\ \ \\ \ 
\end{frame}

\begin{frame}{A function}
  \texttt{myOperation ::\ RealWorld -> (Int, RealWorld)
    myOperation w0 = \\
    let (a,w1) = readNumber(w0) in \\
    \ \ let (b,w2) = readNumber(w1) in \\
    \ \ \ \ (a*2 + 3*b, w2)
  } \\ \ \\ \ \\ \url{https://wiki.haskell.org/IO_inside}
\end{frame}

\begin{frame}{Downsides}
  \pause
  \begin{itemize}
  \item need to pass additional parameter \pause
  \item prone to errors (e.g. \texttt{w0} instead of \texttt{w1}) \pause
  \item nesting level increases \pause
  \end{itemize} \ \\ Could we do something to make \texttt{w0} passed implicitly?
\end{frame}


\begin{frame}{How could this look like?}
  \texttt{pass readNumber \\
    \ \ \ \ \ (λ a \pause -> pass readNumber \\
    \ \ \ \ \ \ \ \ \ \ \ \ \ \ \ \ \ \ (λ b \pause -> give a*2 + 3*b)) \\
    \ \\ \pause
    give value world = (value, world) \\ \pause
    \ \\
    pass value continuation w0 = \\
    \ \ let (result, w1) = value w0 in \\
    \ \ \ \ \ continuation result w1
  }
\end{frame}

\begin{frame}{Does this really work?}
  \texttt{pass readNumber\\
    \ \ \ \ \ (λ a -> pass readNumber\\
    \ \ \ \ \ \ \ \ \ \ \ \ \ \ \ \ \ \ (λ b -> \textbf{give a*2 + 3*b}))\\
    \ \\ \pause
    \textbf{give value world = (value, world)} \\ \pause
    \ \\
    pass value continuation w0 = \\
    \ \ let (result, w1) = value w0 in \\
    \ \ \ \ \ continuation result w1
  }
\end{frame}

\begin{frame}{Does this really work?}
  \texttt{pass readNumber\\
    \ \ \ \ \ (λ a -> pass readNumber\\
    \ \ \ \ \ \ \ \ \ \ \ \ \ \ (λ b -> \textbf{λ w -> (a*2 + 3*b, w)}))\\
    \ \\
    \textbf{give value world = (value, world)} \\
    \ \\
    pass value continuation w0 = \\
    \ \ let (result, w1) = value w0 in \\
    \ \ \ \ \ continuation result w1
  }
\end{frame}

\begin{frame}{Does this really work?}
  \texttt{pass readNumber\\
    \ \ \ \ \ (λ a -> \textbf{pass readNumber\\
      \ \ \ \ \ \ \ \ \ \ \ \ \ \ (λ b -> λ w -> (a*2 + 3*b, w))})\\
    \ \\
    give value world = (value, world) \\
    \ \\
    \textbf{pass value continuation w0 = \\
      \ \ let (result, w1) = value w0 in \\
      \ \ \ \ \ continuation result w1}
  }
\end{frame}

\begin{frame}{Does this really work?}
  \texttt{pass readNumber (λ a -> \textbf{λ w1 -> \\
      \ \ let (y, w2) = readNumber(w1) in\\
      \ \ \ \ \ \ \ \ (λ b -> λ w -> (a*2 + 3*b, w)) y w2})\\
    \ \\
    give value world = (value, world) \\
    \ \\
    \textbf{pass value continuation w0 = \\
      \ \ let (result, w1) = value w0 in \\
      \ \ \ \ \ continuation result w1}
  }
\end{frame}

\begin{frame}{Does this really work?}
  \texttt{pass readNumber (λ a -> λ w1 -> \\
    \ \ let (y, w2) = readNumber(w1) in\\
    \ \ \ \ \ \ \ \ \ \textbf{(λ b -> λ w -> (a*2 + 3*b, w)) y w2})\\
    \ \\
    give value world = (value, world) \\
    \ \\
    pass value continuation w0 = \\
    \ \ let (result, w1) = value w0 in \\
    \ \ \ \ \ continuation result w1
  }
\end{frame}

\begin{frame}{Does this really work?}
  \texttt{pass readNumber (λ a -> λ w1 -> \\
    \ \ let (y, w2) = readNumber(w1) in\\
    \ \ \ \ \ \ \ \ \ \textbf{(λ w -> (a*2 + 3*y, w)) w2})\\
    \ \\
    give value world = (value, world) \\
    \ \\
    pass value continuation w0 = \\
    \ \ let (result, w1) = value w0 in \\
    \ \ \ \ \ continuation result w1
  }
\end{frame}

\begin{frame}{Does this really work?}
  \texttt{pass readNumber (λ a -> λ w1 -> \\
    \ \ let (y, w2) = readNumber(w1) in\\
    \ \ \ \ \ \ \ \ \ \textbf{(a*2 + 3*y, w2)})\\
    \ \\
    give value world = (value, world) \\
    \ \\
    pass value continuation w0 = \\
    \ \ let (result, w1) = value w0 in \\
    \ \ \ \ \ continuation result w1
  }
\end{frame}

\begin{frame}{Does this really work?}
  \texttt{\textbf{pass readNumber (λ a -> λ w1 ->\\
      \ \ let (y, w2) = readNumber(w1) in\\
      \ \ \ \ \ \ \ \ (a*2 + 3*y, w2))}\\
    \ \\
    give value world = (value, world) \\
    \ \\
    \textbf{pass value continuation w0 = \\
      \ \ let (result, w1) = value w0 in \\
      \ \ \ \ \ continuation result w1}
  }
\end{frame}

\begin{frame}{Does this really work?}
  \texttt{\textbf{λ w0 -> let (x,w3) = readNumber(w0) in\\
      \ \ (λ a -> λ w1 -> let (y, w2) = readNumber(w1) \\
      \ \ \ \ \ \ in (a*2 + 3*y, w2)) x w3}\\
    \ \\
    give value world = (value, world) \\
    \ \\
    \textbf{pass value continuation w0 = \\
      \ \ let (result, w1) = value w0 in \\
      \ \ \ \ \ continuation result w1}
  }
\end{frame}

\begin{frame}{Does this really work?}
  \texttt{λ w0 -> let (x,w3) = readNumber(w0) in\\
    \ \ \textbf{(λ a -> λ w1 -> let (y, w2) = readNumber(w1)\\
      \ \ \ \ \ \ in (a*2 + 3*y, w2)) x w3}\\
    \ \\
    give value world = (value, world) \\
    \ \\
    pass value continuation w0 = \\
    \ \ let (result, w1) = value w0 in \\
    \ \ \ \ \ continuation result w1
  }
\end{frame}

\begin{frame}{Does this really work?}
  \texttt{λ w0 -> let (x,w3) = readNumber(w0) in\\
    \ \ \textbf{(λ w1 -> let (y, w2) = readNumber(w1) in\\
      \ \ \ \ \ \ \ \ \ (x*2 + 3*y, w2)) w3}\\
    \ \\
    give value world = (value, world) \\
    \ \\
    pass value continuation w0 = \\
    \ \ let (result, w1) = value w0 in \\
    \ \ \ \ \ continuation result w1
  }
\end{frame}

\begin{frame}{Does this really work?}
  \texttt{λ w0 -> let (x,w3) = readNumber(w0) in\\
    \ \ \textbf{let (y, w2) = readNumber(w3) in\\
      \ \ \ \ \ \ \ \ \ (x*2 + 3*y, w2)}\\
    \ \\
    give value world = (value, world) \\
    \ \\
    pass value continuation w0 = \\
    \ \ let (result, w1) = value w0 in \\
    \ \ \ \ \ continuation result w1
  }
\end{frame}


\begin{frame}{It works!}
  \texttt{pass readNumber \\
    \ \ \ \ \ (λ a -> pass readNumber \\
    \ \ \ \ \ \ \ \ \ \ \ \ \ \ \ \ \ \ (λ b -> give a*2 + 3*b))
  } \\ \pause
  But typing λ and the increased indentation level is annoying! \\ \pause
  \ \\ Introduce new syntax (\texttt{do}-notation): \\
  \texttt{
    \ \ do x <- action1 \\
    \ \ \ \ \ \ action2 \\
    \ \ \ \ \ \ ...
  } \ \\ \pause
  can be interpreted as: \\ \pause
  \texttt{
    \ \ pass action1 (λ x -> do action2 ...)
  }
\end{frame}

\begin{frame}{Caveat emptor}
  In Haskell, \texttt{pass} function is spelled \texttt{>>=} and pronounced ``bind''.
  The \texttt{give} function is called \texttt{return}.
\end{frame}


\begin{frame}{Emperor's new clothes}
  Now we can write our program as:
  \texttt{
    do a <- readNumber
       b <- readNumber
       return a*2 + 3*b
  }
\end{frame}



\end{document}
