\begin{frame}{Podzielniki liczby}
  Liczba $1 \le k \le n$ jest \emph{podzielnikiem} liczby $n$ wtedy i tylko wtedy,
  gdy reszta z dzielenia $n$ przez $k$ wynosi zero\\
  \texttt{
(define (divisors n)\\*
\ \ (define (divisors n k)\\*
\ \ \ \ (if (> k n)\\*
\ \ \ \ \ \ '()\\*
\ \ \ \ \ \ (if (equal?\ (remainder n k) 0)\\*
\ \ \ \ \ \ \ \ (cons k (divisors n (+ k 1)))\\*
\ \ \ \ \ \ \ \ (divisors n (+ k 1)))))\\*
\ \ (divisors n 1))
\ \\*
\ \\*
  }
\end{frame}

\begin{frame}{Podzielniki liczby}
  Liczba $1 \le k \le n$ jest podzielnikiem liczby $n$ wtedy i tylko wtedy,
  gdy reszta z dzielenia $n$ przez $k$ wynosi zero\\
  \texttt{
(define (divisors n)\\*
\ \ \textbf{(define (divisors n k)\\*
\ \ \ \ (if (> k n)\\*
\ \ \ \ \ \ '()\\*
\ \ \ \ \ \ (if (equal?\ (remainder n k) 0)\\*
\ \ \ \ \ \ \ \ (cons k (divisors n (+ k 1)))\\*
\ \ \ \ \ \ \ \ (divisors n (+ k 1)))))}\\*
\ \ (divisors n 1))
\ \\*
\ \\*
  }
\end{frame}

\begin{frame}{Podzielniki liczby}
  Liczba $1 \le k \le n$ jest podzielnikiem liczby $n$ wtedy i tylko wtedy,
  gdy reszta z dzielenia $n$ przez $k$ wynosi zero\\
  \texttt{
(define (divisors n)\\*
\ \ (define (divisors n k)\\*
\ \ \ \ (if \textbf{(> k n)\\*
\ \ \ \ \ \ '()}\\*
\ \ \ \ \ \ (if (equal?\ (remainder n k) 0)\\*
\ \ \ \ \ \ \ \ (cons k (divisors n (+ k 1)))\\*
\ \ \ \ \ \ \ \ (divisors n (+ k 1)))))\\*
\ \ (divisors n 1)
\ \\*
\ \\*
  }
\end{frame}

\begin{frame}{Podzielniki liczby}
  Liczba $1 \le k \le n$ jest podzielnikiem liczby $n$ wtedy i tylko wtedy,
  gdy reszta z dzielenia $n$ przez $k$ wynosi zero\\
  \texttt{
(define (divisors n)\\*
\ \ (define (divisors n k)\\*
\ \ \ \ (if (> k n)\\*
\ \ \ \ \ \ '()\\*
\ \ \ \ \ \ \textbf{(if (equal?\ (remainder n k) 0)\\*
\ \ \ \ \ \ \ \ (cons k (divisors n (+ k 1)))\\*
\ \ \ \ \ \ \ \ (divisors n (+ k 1)))}))\\*
\ \ (divisors n 1))
\ \\*
\ \\*
  }
\end{frame}

\begin{frame}{Podzielniki liczby}
  Liczba $1 \le k \le n$ jest podzielnikiem liczby $n$ wtedy i tylko wtedy,
  gdy reszta z dzielenia $n$ przez $k$ wynosi zero\\
  \texttt{
(define (divisors n)\\*
\ \ (define (divisors n k)\\*
\ \ \ \ (if (> k n)\\*
\ \ \ \ \ \ '()\\*
\ \ \ \ \ \ (if \textbf{(equal?\ (remainder n k) 0)\\*
\ \ \ \ \ \ \ \ (cons k (divisors n (+ k 1)))}\\*
\ \ \ \ \ \ \ \ (divisors n (+ k 1)))))\\*
\ \ (divisors n 1))
\ \\*
\ \\*
  }
\end{frame}

\begin{frame}{Podzielniki liczby}
  Liczba $1 \le k \le n$ jest podzielnikiem liczby $n$ wtedy i tylko wtedy,
  gdy reszta z dzielenia $n$ przez $k$ wynosi zero\\
  \texttt{
(define (divisors n)\\*
\ \ (define (divisors n k)\\*
\ \ \ \ (if (> k n)\\*
\ \ \ \ \ \ '()\\*
\ \ \ \ \ \ (if (equal?\ (remainder n k) 0)\\*
\ \ \ \ \ \ \ \ (cons k (divisors n (+ k 1)))\\*
\ \ \ \ \ \ \ \ \textbf{(divisors n (+ k 1))})))\\*
\ \ (divisors n 1))
\ \\*
\ \\*
  }
\end{frame}

\begin{frame}{Podzielniki liczby}
  Liczba $1 \le k \le n$ jest podzielnikiem liczby $n$ wtedy i tylko wtedy,
  gdy reszta z dzielenia $n$ przez $k$ wynosi zero\\
  \texttt{
(define (divisors n)\\*
\ \ (define (divisors n k)\\*
\ \ \ \ (if (> k n)\\*
\ \ \ \ \ \ '()\\*
\ \ \ \ \ \ (if (equal?\ (remainder n k) 0)\\*
\ \ \ \ \ \ \ \ (cons k (divisors n (+ k 1)))\\*
\ \ \ \ \ \ \ \ (divisors n (+ k 1)))))\\*
\ \ \textbf{(divisors n 1)})
\ \\*
\ \\*
  }
\end{frame}

\begin{frame}{Analiza}
  Rozważmy \texttt{(divisors 4)}. Na mocy definicji mamy\pause\\
  \texttt{(divisors 4 1)}\pause\\
  \texttt{
(if (> 1 4)\\*
\ \ '()\\*
\ \ (if (equal?\ (remainder 4 1) 0)\\*
\ \ \ \ (cons 1 (divisors 4 (+ 1 1)))\\*
\ \ \ \ (divisors 4 (+ 1 1))))\\*
\ \\*
\ \\*
}
\end{frame}


\begin{frame}{Analiza}
  Rozważmy \texttt{(divisors 4)}. Na mocy definicji mamy\\
  \texttt{(divisors 4 1)}\\
  \texttt{
(if \textbf{(> 1 4)}\\*
\ \ '()\\*
\ \ (if (equal?\ (remainder 4 1) 0)\\*
\ \ \ \ (cons 1 (divisors 4 (+ 1 1)))\\*
\ \ \ \ (divisors 4 (+ 1 1))))\\*
\ \\*
\ \\*
}
\end{frame}

\begin{frame}{Analiza}
  Rozważmy \texttt{(divisors 4)}. Na mocy definicji mamy\\
  \texttt{(divisors 4 1)}\\
  \texttt{
(if \#false\\*
\ \ '()\\*
\ \ \textbf{(if (equal?\ (remainder 4 1) 0)\\*
\ \ \ \ (cons 1 (divisors 4 (+ 1 1)))\\*
\ \ \ \ (divisors 4 (+ 1 1)))})\\*
\ \\*
\ \\*
}
\end{frame}

\begin{frame}{Analiza}
  Rozważmy \texttt{(divisors 4)}. Na mocy definicji mamy\\
  \texttt{(divisors 4 1)}\\
  \texttt{
\textbf{(if (equal?\ (remainder 4 1) 0)\\*
\ \ (cons 1 (divisors 4 (+ 1 1)))\\*
\ \ (divisors 4 (+ 1 1)))}\\*
\ \\*
\ \\*
\ \\*
\ \\*
}
\end{frame}

\begin{frame}{Analiza}
  Rozważmy \texttt{(divisors 4)}. Na mocy definicji mamy\\
  \texttt{(divisors 4 1)}\\
  \texttt{
(if \textbf{(equal?\ (remainder 4 1) 0)}\\*
\ \ (cons 1 (divisors 4 (+ 1 1)))\\*
\ \ (divisors 4 (+ 1 1)))\\*
\ \\*
\ \\*
\ \\*
\ \\*
}
\end{frame}

\begin{frame}{Analiza}
  Rozważmy \texttt{(divisors 4)}. Na mocy definicji mamy\\
  \texttt{(divisors 4 1)}\\
  \texttt{
(if (equal?\ \textbf{(remainder 4 1)} 0)\\*
\ \ (cons 1 (divisors 4 (+ 1 1)))\\*
\ \ (divisors 4 (+ 1 1)))\\*
\ \\*
\ \\*
\ \\*
\ \\*
}
\end{frame}

\begin{frame}{Analiza}
  Rozważmy \texttt{(divisors 4)}. Na mocy definicji mamy\\
  \texttt{(divisors 4 1)}\\
  \texttt{
(if (equal?\ \textbf{0} 0)\\*
\ \ (cons 1 (divisors 4 (+ 1 1)))\\*
\ \ (divisors 4 (+ 1 1)))\\*
\ \\*
\ \\*
\ \\*
\ \\*
}
\end{frame}

\begin{frame}{Analiza}
  Rozważmy \texttt{(divisors 4)}. Na mocy definicji mamy\\
  \texttt{(divisors 4 1)}\\
  \texttt{
(if \textbf{(equal?\ 0 0)}\\*
\ \ (cons 1 (divisors 4 (+ 1 1)))\\*
\ \ (divisors 4 (+ 1 1)))\\*
\ \\*
\ \\*
\ \\*
\ \\*
}
\end{frame}

\begin{frame}{Analiza}
  Rozważmy \texttt{(divisors 4)}. Na mocy definicji mamy\\
  \texttt{(divisors 4 1)}\\
  \texttt{
(if \#true\\*
\ \ \textbf{(cons 1 (divisors 4 (+ 1 1)))}\\*
\ \ (divisors 4 (+ 1 1)))\\*
\ \\*
\ \\*
\ \\*
\ \\*
}
\end{frame}


\begin{frame}{Analiza}
  Rozważmy \texttt{(divisors 4)}. Na mocy definicji mamy\\
  \texttt{(divisors 4 1)}\\
  \texttt{
(cons 1 \textbf{(divisors 4 (+ 1 1))})\\*
\ \\*    
\ \\*
\ \\*
\ \\*
\ \\*
\ \\*
  }
\end{frame}

\begin{frame}{Analiza}
  Rozważmy \texttt{(divisors 4)}. Na mocy definicji mamy\\
  \texttt{(divisors 4 1)}\\
  \texttt{
(cons 1 (divisors 4 \textbf{(+ 1 1)}))\\*
\ \\*    
\ \\*
\ \\*
\ \\*
\ \\*
\ \\*
  }
\end{frame}

\begin{frame}{Analiza}
  Rozważmy \texttt{(divisors 4)}. Na mocy definicji mamy\\
  \texttt{(divisors 4 1)}\\
  \texttt{
(cons 1 (divisors 4 \textbf{2}))\\*
\ \\*    
\ \\*
\ \\*
\ \\*
\ \\*
\ \\*
  }
\end{frame}

\begin{frame}{Analiza}
  Rozważmy \texttt{(divisors 4)}. Na mocy definicji mamy\\
  \texttt{(divisors 4 1)}\\
  \texttt{
(cons 1 \textbf{(divisors 4 2)})\\*
\ \\*
\ \\*
\ \\*
\ \\*
\ \\*
\ \\*
  }
\end{frame}

\begin{frame}{Analiza}
  Rozważmy \texttt{(divisors 4)}. Na mocy definicji mamy\\
  \texttt{(divisors 4 1)}\\
  \texttt{
(cons 1 \textbf{(if (> 2 4)\\*
\ \ \ \ \ \ \ \ \ \ \ '()\\*
\ \ \ \ \ \ \ \ \ \ \ (if (equal?\ (remainder 4 2) 0)\\*
\ \ \ \ \ \ \ \ \ \ \ \ \ (cons 2 (divisors 4 (+ 2 1)))\\*
\ \ \ \ \ \ \ \ \ \ \ \ \ (divisors 4 (+ 2 1))))})
\ \\*
\ \\*
\ \\*
  }
\end{frame}


\begin{frame}{Analiza}
  Rozważmy \texttt{(divisors 4)}. Na mocy definicji mamy\\
  \texttt{(divisors 4 1)}\\
  \texttt{
(cons 1 (if \textbf{(> 2 4)}\\*
\ \ \ \ \ \ \ \ \ \ \ '()\\*
\ \ \ \ \ \ \ \ \ \ \ (if (equal?\ (remainder 4 2) 0)\\*
\ \ \ \ \ \ \ \ \ \ \ \ \ (cons 2 (divisors 4 (+ 2 1)))\\*
\ \ \ \ \ \ \ \ \ \ \ \ \ (divisors 4 (+ 2 1)))))
\ \\*
\ \\*
\ \\*
  }
\end{frame}

\begin{frame}{Analiza}
  Rozważmy \texttt{(divisors 4)}. Na mocy definicji mamy\\
  \texttt{(divisors 4 1)}\\
  \texttt{
(cons 1 (if \#false\\*
\ \ \ \ \ \ \ \ \ \ \ '()\\*
\ \ \ \ \ \ \ \ \ \ \ \textbf{(if (equal?\ (remainder 4 2) 0)\\*
\ \ \ \ \ \ \ \ \ \ \ \ \ (cons 2 (divisors 4 (+ 2 1)))\\*
\ \ \ \ \ \ \ \ \ \ \ \ \ (divisors 4 (+ 2 1)))}))
\ \\*
\ \\*
\ \\*
  }
\end{frame}

\begin{frame}{Analiza}
  Rozważmy \texttt{(divisors 4)}. Na mocy definicji mamy\\
  \texttt{(divisors 4 1)}\\
  \texttt{
(cons 1 \textbf{(if (equal?\ (remainder 4 2) 0)\\*
\ \ \ \ \ \ \ \ \ \ \ (cons 2 (divisors 4 (+ 2 1)))\\*
\ \ \ \ \ \ \ \ \ \ \ (divisors 4 (+ 2 1)))})\\*
    \ \\*
    \ \\*
    \ \\*
    \ \\*
  }
\end{frame}

\begin{frame}{Analiza}
  Rozważmy \texttt{(divisors 4)}. Na mocy definicji mamy\\
  \texttt{(divisors 4 1)}\\
  \texttt{
(cons 1 (if \textbf{(equal?\ (remainder 4 2) 0)}\\*
\ \ \ \ \ \ \ \ \ \ \ (cons 2 (divisors 4 (+ 2 1)))\\*
\ \ \ \ \ \ \ \ \ \ \ (divisors 4 (+ 2 1))))\\*
    \ \\*
    \ \\*
    \ \\*
    \ \\*
  }
\end{frame}

\begin{frame}{Analiza}
  Rozważmy \texttt{(divisors 4)}. Na mocy definicji mamy\\
  \texttt{(divisors 4 1)}\\
  \texttt{
(cons 1 (if (equal?\ \textbf{(remainder 4 2)} 0)\\*
\ \ \ \ \ \ \ \ \ \ \ (cons 2 (divisors 4 (+ 2 1)))\\*
\ \ \ \ \ \ \ \ \ \ \ (divisors 4 (+ 2 1))))\\*
    \ \\*
    \ \\*
    \ \\*
    \ \\*
  }
\end{frame}

\begin{frame}{Analiza}
  Rozważmy \texttt{(divisors 4)}. Na mocy definicji mamy\\
  \texttt{(divisors 4 1)}\\
  \texttt{
(cons 1 (if (equal?\ \textbf{0} 0)\\*
\ \ \ \ \ \ \ \ \ \ \ (cons 2 (divisors 4 (+ 2 1)))\\*
\ \ \ \ \ \ \ \ \ \ \ (divisors 4 (+ 2 1))))\\*
    \ \\*
    \ \\*
    \ \\*
    \ \\*
  }
\end{frame}

\begin{frame}{Analiza}
  Rozważmy \texttt{(divisors 4)}. Na mocy definicji mamy\\
  \texttt{(divisors 4 1)}\\
  \texttt{
(cons 1 (if \textbf{(equal?\ 0 0)}\\*
\ \ \ \ \ \ \ \ \ \ \ (cons 2 (divisors 4 (+ 2 1)))\\*
\ \ \ \ \ \ \ \ \ \ \ (divisors 4 (+ 2 1))))\\*
    \ \\*
    \ \\*
    \ \\*
    \ \\*
  }
\end{frame}

\begin{frame}{Analiza}
  Rozważmy \texttt{(divisors 4)}. Na mocy definicji mamy\\
  \texttt{(divisors 4 1)}\\
  \texttt{
(cons 1 (if \#true\\*
\ \ \ \ \ \ \ \ \ \ \ \textbf{(cons 2 (divisors 4 (+ 2 1)))}\\*
\ \ \ \ \ \ \ \ \ \ \ (divisors 4 (+ 2 1))))\\*
    \ \\*
    \ \\*
    \ \\*
    \ \\*
  }
\end{frame}

\begin{frame}{Analiza}
  Rozważmy \texttt{(divisors 4)}. Na mocy definicji mamy\\
  \texttt{(divisors 4 1)}\\
  \texttt{
(cons 1 \textbf{(cons 2 (divisors 4 (+ 2 1)))})\\*
    \ \\*
    \ \\*
    \ \\*
    \ \\*
    \ \\*
    \ \\*
  }
\end{frame}


\begin{frame}{Analiza}
  Rozważmy \texttt{(divisors 4)}. Na mocy definicji mamy\\
  \texttt{(divisors 4 1)}\\
  \texttt{
(cons 1 (cons 2 (divisors 4 \textbf{(+ 2 1)})))\\*
    \ \\*
    \ \\*
    \ \\*
    \ \\*
    \ \\*
    \ \\*
  }
\end{frame}

\begin{frame}{Analiza}
  Rozważmy \texttt{(divisors 4)}. Na mocy definicji mamy\\
  \texttt{(divisors 4 1)}\\
  \texttt{
(cons 1 (cons 2 (divisors 4 \textbf{3})))\\*
    \ \\*
    \ \\*
    \ \\*
    \ \\*
    \ \\*
    \ \\*
  }
\end{frame}

\begin{frame}{Analiza}
  Rozważmy \texttt{(divisors 4)}. Na mocy definicji mamy\\
  \texttt{(divisors 4 1)}\\
  \texttt{
(cons 1 \\*
\ \ (cons 2 \\*
\ \ \ \ \textbf{(divisors 4 3)}))\\*
    \ \\*
    \ \\*
    \ \\*
    \ \\*
  }
\end{frame}

\begin{frame}{Analiza}
  Rozważmy \texttt{(divisors 4)}. Na mocy definicji mamy\\
  \texttt{(divisors 4 1)}\\
  \texttt{
(cons 1 \\*
\ \ (cons 2 \\*
\ \ \ \ \textbf{(if (> 3 4)\\*
\ \ \ \ \ \ '()\\*
\ \ \ \ \ \ (if (equal?\ (remainder 4 3) 0)\\*
\ \ \ \ \ \ \ \ (cons 3 (divisors 4 (+ 4 1)))\\*
\ \ \ \ \ \ \ \ (divisors 4 (+ 3 1))))}))\\*
  }
\end{frame}

\begin{frame}{Analiza}
  Rozważmy \texttt{(divisors 4)}. Na mocy definicji mamy\\
  \texttt{(divisors 4 1)}\\
  \texttt{
(cons 1 \\*
\ \ (cons 2 \\*
\ \ \ \ (if \textbf{(> 3 4)}\\*
\ \ \ \ \ \ '()\\*
\ \ \ \ \ \ (if (equal?\ (remainder 4 3) 0)\\*
\ \ \ \ \ \ \ \ (cons 3 (divisors 4 (+ 4 1)))\\*
\ \ \ \ \ \ \ \ (divisors 4 (+ 3 1))))))\\*
  }
\end{frame}


\begin{frame}{Analiza}
  Rozważmy \texttt{(divisors 4)}. Na mocy definicji mamy\\
  \texttt{(divisors 4 1)}\\
  \texttt{
(cons 1 \\*
\ \ (cons 2 \\*
\ \ \ \ (if \#false\\*
\ \ \ \ \ \ '()\\*
\ \ \ \ \ \ \textbf{(if (equal?\ (remainder 4 3) 0)\\*
\ \ \ \ \ \ \ \ (cons 3 (divisors 4 (+ 4 1)))\\*
\ \ \ \ \ \ \ \ (divisors 4 (+ 3 1)))})))\\*
  }
\end{frame}


\begin{frame}{Analiza}
  Rozważmy \texttt{(divisors 4)}. Na mocy definicji mamy\\
  \texttt{(divisors 4 1)}\\
  \texttt{
(cons 1 \\*
\ \ (cons 2 \\*
\ \ \ \ \textbf{(if (equal?\ (remainder 4 3) 0)\\*
\ \ \ \ \ \ (cons 3 (divisors 4 (+ 4 1)))\\*
\ \ \ \ \ \ (divisors 4 (+ 3 1)))}))\\*
\ \\*
\ \\*
  }
\end{frame}

\begin{frame}{Analiza}
  Rozważmy \texttt{(divisors 4)}. Na mocy definicji mamy\\
  \texttt{(divisors 4 1)}\\
  \texttt{
(cons 1 \\*
\ \ (cons 2 \\*
\ \ \ \ (if \textbf{(equal?\ (remainder 4 3) 0)}\\*
\ \ \ \ \ \ (cons 3 (divisors 4 (+ 4 1)))\\*
\ \ \ \ \ \ (divisors 4 (+ 3 1)))))\\*
\ \\*
\ \\*
  }
\end{frame}

\begin{frame}{Analiza}
  Rozważmy \texttt{(divisors 4)}. Na mocy definicji mamy\\
  \texttt{(divisors 4 1)}\\
  \texttt{
(cons 1 \\*
\ \ (cons 2 \\*
\ \ \ \ (if \textbf{(equal?\ (remainder 4 3) 0)}\\*
\ \ \ \ \ \ (cons 3 (divisors 4 (+ 4 1)))\\*
\ \ \ \ \ \ (divisors 4 (+ 3 1)))))\\*
\ \\*
\ \\*
  }
\end{frame}

\begin{frame}{Analiza}
  Rozważmy \texttt{(divisors 4)}. Na mocy definicji mamy\\
  \texttt{(divisors 4 1)}\\
  \texttt{
(cons 1 \\*
\ \ (cons 2 \\*
\ \ \ \ (if (equal?\ \textbf{(remainder 4 3)} 0)\\*
\ \ \ \ \ \ (cons 3 (divisors 4 (+ 4 1)))\\*
\ \ \ \ \ \ (divisors 4 (+ 3 1)))))\\*
\ \\*
\ \\*
  }
\end{frame}


\begin{frame}{Analiza}
  Rozważmy \texttt{(divisors 4)}. Na mocy definicji mamy\\
  \texttt{(divisors 4 1)}\\
  \texttt{
(cons 1 \\*
\ \ (cons 2 \\*
\ \ \ \ (if (equal?\ \textbf{1} 0)\\*
\ \ \ \ \ \ (cons 3 (divisors 4 (+ 4 1)))\\*
\ \ \ \ \ \ (divisors 4 (+ 3 1)))))\\*
\ \\*
\ \\*
  }
\end{frame}


\begin{frame}{Analiza}
  Rozważmy \texttt{(divisors 4)}. Na mocy definicji mamy\\
  \texttt{(divisors 4 1)}\\
  \texttt{
(cons 1 \\*
\ \ (cons 2 \\*
\ \ \ \ (if \texttt{(equal?\ 1 0)}\\*
\ \ \ \ \ \ (cons 3 (divisors 4 (+ 4 1)))\\*
\ \ \ \ \ \ (divisors 4 (+ 3 1)))))\\*
\ \\*
\ \\*
  }
\end{frame}

\begin{frame}{Analiza}
  Rozważmy \texttt{(divisors 4)}. Na mocy definicji mamy\\
  \texttt{(divisors 4 1)}\\
  \texttt{
(cons 1 \\*
\ \ (cons 2 \\*
\ \ \ \ (if \#false\\*
\ \ \ \ \ \ (cons 3 (divisors 4 (+ 4 1)))\\*
\ \ \ \ \ \ \textbf{(divisors 4 (+ 3 1))})))\\*
\ \\*
\ \\*
  }
\end{frame}

\begin{frame}{Analiza}
  Rozważmy \texttt{(divisors 4)}. Na mocy definicji mamy\\
  \texttt{(divisors 4 1)}\\
  \texttt{
(cons 1 \\*
\ \ (cons 2 \\*
\ \ \ \ \textbf{(divisors 4 (+ 3 1))}))\\*
\ \\*
\ \\*
\ \\*
\ \\*
  }
\end{frame}

\begin{frame}{Analiza}
  Rozważmy \texttt{(divisors 4)}. Na mocy definicji mamy\\
  \texttt{(divisors 4 1)}\\
  \texttt{
(cons 1 \\*
\ \ (cons 2 \\*
\ \ \ \ (divisors 4 \textbf{(+ 3 1)})))\\*
\ \\*
\ \\*
\ \\*
\ \\*
  }
\end{frame}

\begin{frame}{Analiza}
  Rozważmy \texttt{(divisors 4)}. Na mocy definicji mamy\\
  \texttt{(divisors 4 1)}\\
  \texttt{
(cons 1 \\*
\ \ (cons 2 \\*
\ \ \ \ (divisors 4 \textbf{4})))\\*
\ \\*
\ \\*
\ \\*
\ \\*
  }
\end{frame}

\begin{frame}{Analiza}
  Rozważmy \texttt{(divisors 4)}. Na mocy definicji mamy\\
  \texttt{(divisors 4 1)}\\
  \texttt{
(cons 1 \\*
\ \ (cons 2 \\*
\ \ \ \ \textbf{(divisors 4 4)}))\\*
\ \\*
\ \\*
\ \\*
\ \\*
  }
\end{frame}


\begin{frame}{Analiza}
  Rozważmy \texttt{(divisors 4)}. Na mocy definicji mamy\\
  \texttt{(divisors 4 1)}\\
  \texttt{
(cons 1 \\*
\ \ (cons 2 \\*
\ \ \ \ \textbf{(if (> 4 4)\\*
\ \ \ \ \ \ '()\\*
\ \ \ \ \ \ (if (equal?\ (remainder 4 4) 0)\\*
\ \ \ \ \ \ \ \ (cons 4 (divisors 4 (+ 4 1)))\\*
\ \ \ \ \ \ \ \ (divisors 4 (+ 4 1))))}))\\*
  }
\end{frame}

\begin{frame}{Analiza}
  Rozważmy \texttt{(divisors 4)}. Na mocy definicji mamy\\
  \texttt{(divisors 4 1)}\\
  \texttt{
(cons 1 \\*
\ \ (cons 2 \\*
\ \ \ \ (if \textbf{(> 4 4)}\\*
\ \ \ \ \ \ '()\\*
\ \ \ \ \ \ (if (equal?\ (remainder 4 4) 0)\\*
\ \ \ \ \ \ \ \ (cons 4 (divisors 4 (+ 4 1)))\\*
\ \ \ \ \ \ \ \ (divisors 4 (+ 4 1))))))\\*
  }
\end{frame}

\begin{frame}{Analiza}
  Rozważmy \texttt{(divisors 4)}. Na mocy definicji mamy\\
  \texttt{(divisors 4 1)}\\
  \texttt{
(cons 1 \\*
\ \ (cons 2 \\*
\ \ \ \ (if \#false\\*
\ \ \ \ \ \ '()\\*
\ \ \ \ \ \ \textbf{(if (equal?\ (remainder 4 4) 0)\\*
\ \ \ \ \ \ \ \ (cons 4 (divisors 4 (+ 4 1)))\\*
\ \ \ \ \ \ \ \ (divisors 4 (+ 4 1)))})))\\*
  }
\end{frame}


\begin{frame}{Analiza}
  Rozważmy \texttt{(divisors 4)}. Na mocy definicji mamy\\
  \texttt{(divisors 4 1)}\\
  \texttt{
(cons 1 \\*
\ \ (cons 2 \\*
\ \ \ \ (if \textbf{(equal?\ (remainder 4 4) 0)}\\*
\ \ \ \ \ \ (cons 4 (divisors 4 (+ 4 1)))\\*
\ \ \ \ \ \ (divisors 4 (+ 4 1)))))\\*
\ \\*
\ \\*
  }
\end{frame}

\begin{frame}{Analiza}
  Rozważmy \texttt{(divisors 4)}. Na mocy definicji mamy\\
  \texttt{(divisors 4 1)}\\
  \texttt{
(cons 1 \\*
\ \ (cons 2 \\*
\ \ \ \ (if (equal?\ \textbf{(remainder 4 4)} 0)\\*
\ \ \ \ \ \ (cons 4 (divisors 4 (+ 4 1)))\\*
\ \ \ \ \ \ (divisors 4 (+ 4 1)))))\\*
\ \\*
\ \\*
  }
\end{frame}

\begin{frame}{Analiza}
  Rozważmy \texttt{(divisors 4)}. Na mocy definicji mamy\\
  \texttt{(divisors 4 1)}\\
  \texttt{
(cons 1 \\*
\ \ (cons 2 \\*
\ \ \ \ (if (equal?\ \textbf{0} 0)\\*
\ \ \ \ \ \ (cons 4 (divisors 4 (+ 4 1)))\\*
\ \ \ \ \ \ (divisors 4 (+ 4 1)))))\\*
\ \\*
\ \\*
  }
\end{frame}

\begin{frame}{Analiza}
  Rozważmy \texttt{(divisors 4)}. Na mocy definicji mamy\\
  \texttt{(divisors 4 1)}\\
  \texttt{
(cons 1 \\*
\ \ (cons 2 \\*
\ \ \ \ (if \textbf{(equal?\ 0 0)}\\*
\ \ \ \ \ \ (cons 4 (divisors 4 (+ 4 1)))\\*
\ \ \ \ \ \ (divisors 4 (+ 4 1)))))\\*
\ \\*
\ \\*
  }
\end{frame}

\begin{frame}{Analiza}
  Rozważmy \texttt{(divisors 4)}. Na mocy definicji mamy\\
  \texttt{(divisors 4 1)}\\
  \texttt{
(cons 1 \\*
\ \ (cons 2 \\*
\ \ \ \ (if \#true\\*
\ \ \ \ \ \ \textbf{(cons 4 (divisors 4 (+ 4 1)))}\\*
\ \ \ \ \ \ (divisors 4 (+ 4 1)))))\\*
\ \\*
\ \\*
  }
\end{frame}

\begin{frame}{Analiza}
  Rozważmy \texttt{(divisors 4)}. Na mocy definicji mamy\\
  \texttt{(divisors 4 1)}\\
  \texttt{
(cons 1 \\*
\ \ (cons 2 \\*
\ \ \ \ \textbf{(cons 4 (divisors 4 (+ 4 1)))}))\\*
\ \\*
\ \\*
\ \\*
\ \\*
  }
\end{frame}

\begin{frame}{Analiza}
  Rozważmy \texttt{(divisors 4)}. Na mocy definicji mamy\\
  \texttt{(divisors 4 1)}\\
  \texttt{
(cons 1 \\*
\ \ (cons 2 \\*
\ \ \ \ (cons 4 (divisors 4 \textbf{(+ 4 1)}))))\\*
\ \\*
\ \\*
\ \\*
\ \\*
  }
\end{frame}

\begin{frame}{Analiza}
  Rozważmy \texttt{(divisors 4)}. Na mocy definicji mamy\\
  \texttt{(divisors 4 1)}\\
  \texttt{
(cons 1 \\*
\ \ (cons 2 \\*
\ \ \ \ (cons 4 (divisors 4 \textbf{5}))))\\*
\ \\*
\ \\*
\ \\*
\ \\*
  }
\end{frame}

\begin{frame}{Analiza}
  Rozważmy \texttt{(divisors 4)}. Na mocy definicji mamy\\
  \texttt{(divisors 4 1)}\\
  \texttt{
(cons 1 \\*
\ \ (cons 2 \\*
\ \ \ \ (cons 4 \textbf{(divisors 4 5)})))\\*
\ \\*
\ \\*
\ \\*
\ \\*
  }
\end{frame}

\begin{frame}{Analiza}
  Rozważmy \texttt{(divisors 4)}. Na mocy definicji mamy\\
  \texttt{(divisors 4 1)}\\
  \texttt{
(cons 1 \\*
\ \ (cons 2 \\*
\ \ \ \ (cons 4 \textbf{(if (> 5 4)\\*
\ \ \ \ \ \ '()\\*
\ \ \ \ \ \ (if (equal?\ (remainder 4 5) 0)\\*
\ \ \ \ \ \ \ \ (cons 5 (divisors 4 (+ 5 1)))\\*
\ \ \ \ \ \ \ \ (divisors 5 (+ 4 1))))})))\\*
  }
\end{frame}

\begin{frame}{Analiza}
  Rozważmy \texttt{(divisors 4)}. Na mocy definicji mamy\\
  \texttt{(divisors 4 1)}\\
  \texttt{
(cons 1 \\*
\ \ (cons 2 \\*
\ \ \ \ (cons 4 (if \textbf{(> 5 4)}\\*
\ \ \ \ \ \ '()\\*
\ \ \ \ \ \ (if (equal?\ (remainder 4 5) 0)\\*
\ \ \ \ \ \ \ \ (cons 5 (divisors 4 (+ 5 1)))\\*
\ \ \ \ \ \ \ \ (divisors 5 (+ 4 1)))))))\\*
  }
\end{frame}

\begin{frame}{Analiza}
  Rozważmy \texttt{(divisors 4)}. Na mocy definicji mamy\\
  \texttt{(divisors 4 1)}\\
  \texttt{
(cons 1 \\*
\ \ (cons 2 \\*
\ \ \ \ (cons 4 (if \#true\\*
\ \ \ \ \ \ \textbf{'()}\\*
\ \ \ \ \ \ (if (equal?\ (remainder 4 5) 0)\\*
\ \ \ \ \ \ \ \ (cons 5 (divisors 4 (+ 5 1)))\\*
\ \ \ \ \ \ \ \ (divisors 5 (+ 4 1)))))))\\*
  }
\end{frame}

\begin{frame}{Analiza}
  Rozważmy \texttt{(divisors 4)}. Na mocy definicji mamy\\
  \texttt{(divisors 4 1)}\\
  \texttt{
    (cons 1 (cons 2 (cons 4 \textbf{'()})))\\*
    \ \\*
    \ \\*
    \ \\*
    \ \\*
    \ \\*
    \ \\*
  }
\end{frame}

\begin{frame}{Analiza}
  Rozważmy \texttt{(divisors 4)}. Na mocy definicji mamy\\
  \texttt{(divisors 4 1)}\\
  \texttt{
    (cons 1 (cons 2 \textbf{(cons 4 '())}))\\*
    \ \\*
    \ \\*
    \ \\*
    \ \\*
    \ \\*
    \ \\*
  }
\end{frame}

\begin{frame}{Analiza}
  Rozważmy \texttt{(divisors 4)}. Na mocy definicji mamy\\
  \texttt{(divisors 4 1)}\\
  \texttt{
    (cons 1 (cons 2 \textbf{'(4))}))\\*
    \ \\*
    \ \\*
    \ \\*
    \ \\*
    \ \\*
    \ \\*
  }
\end{frame}

\begin{frame}{Analiza}
  Rozważmy \texttt{(divisors 4)}. Na mocy definicji mamy\\
  \texttt{(divisors 4 1)}\\
  \texttt{
    (cons 1 \textbf{(cons 2 '(4))})\\*
    \ \\*
    \ \\*
    \ \\*
    \ \\*
    \ \\*
    \ \\*
  }
\end{frame}

\begin{frame}{Analiza}
  Rozważmy \texttt{(divisors 4)}. Na mocy definicji mamy\\
  \texttt{(divisors 4 1)}\\
  \texttt{
    (cons 1 \textbf{'(2 4)})\\*
    \ \\*
    \ \\*
    \ \\*
    \ \\*
    \ \\*
    \ \\*
  }
\end{frame}

\begin{frame}{Analiza}
  Rozważmy \texttt{(divisors 4)}. Na mocy definicji mamy\\
  \texttt{(divisors 4 1)}\\
  \texttt{
    \textbf{(cons 1 '(2 4))}\\*
    \ \\*
    \ \\*
    \ \\*
    \ \\*
    \ \\*
    \ \\*
  }
\end{frame}

\begin{frame}{Analiza}
  Rozważmy \texttt{(divisors 4)}. Na mocy definicji mamy\\
  \texttt{(divisors 4 1)}\\
  \texttt{
    \textbf{'(1 2 4)}\\*
    \ \\*
    \ \\*
    \ \\*
    \ \\*
    \ \\*
    \ \\*
  }
\end{frame}

\begin{frame}{Analiza}
  Rozważmy \texttt{(divisors 4)}. Na mocy definicji mamy\\
  \texttt{(divisors 4 1)}\\
  \texttt{
    (1 2 4)\\*
    \ \\*
    \ \\*
    \ \\*
    \ \\*
    \ \\*
    \ \\*
  }
\end{frame}
