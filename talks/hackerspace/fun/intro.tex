\documentclass{beamer}

\newenvironment{tightcenter}{%
  \setlength\topsep{0pt}
  \setlength\parskip{0pt}
  \begin{center}
}{%
  \end{center}
}

\mode<presentation>
{
  \usetheme{Copenhagen}
  %%\usecolortheme[RGB={173,222,25}]{structure}
  \usecolortheme[RGB={255,0,0}]{structure}
  \setbeamertemplate{items}[circle]
  \setbeamercovered{transparent}
}

\usepackage[polish]{babel}
\usepackage{chessfss}
\usepackage{hyperref}
\usepackage{qtree}
\usepackage{mathtools}
\usepackage{dirtytalk}
\usepackage{epigraph}
\usepackage{textgreek}
\usepackage[utf8]{inputenc}
\usepackage{times}
\usepackage[T1]{fontenc}
\usepackage{tikz}
\usepackage{csquotes}
\usepackage{amsmath}
\usepackage{fancyvrb}
\usepackage{ulem}
\usepackage{adjustbox}

\newenvironment{Snippet}{\Verbatim[samepage=true]}{\endVerbatim}

\title{\textbf{Czym do diaska jest programowanie funkcyjne?}}

\author{Panicz Maciej Godek}

\institute{
  \tiny{\href{mailto:godek.maciek@gmail.com}{\textbf{godek.maciek@gmail.com}}}
}

\date{\textbf{Hackerspace Trójmiasto}, 07.02.2017}

\begin{document}

\begin{frame}
  \titlepage
\end{frame}

\section{Wprowadzenie}

\begin{frame}{Definicja}
  \textbf{Czym jest programowanie?}
  \pause
  \begin{itemize}
  \item stosunkowo młoda dziedzina ludzkiej działalności
    \pause
    \\ ale czy na pewno?
    \pause
    \begin{displayquote}
      ,,Nauczyciele, generałowie, dietetycy, psychologowie i rodzice programują.
      Działania wojska, studentów i niektórych społeczności są programowane.
      Zmierzenie się z dużym problemem wymaga wielu programów, z których
      większość jest powoływana do życia w trakcie tych zmagań.''
    \end{displayquote}
    -- Alan Perlis
    
  \end{itemize}

\end{frame}

\begin{frame}{Analogia}
  \textbf{Do jakiej innej dziedziny ludzkiej działalności
    można porównać programowanie?}
  \pause
  \begin{itemize}
  \item budowanie urządzeń\pause, które \\
    przeprowadzając kolejne kroki \\
    \pause zmieniają swój stan
    \pause
  \item logika, wyrażanie idei \\
    \pause (definicje, twierdzenia, dowody, przykłady)
  \end{itemize}
\end{frame}


\begin{frame}{Analogia}
  \textbf{Do jakiej innej dziedziny ludzkiej działalności
    można porównać programowanie?}
  \begin{itemize}
  \item budowanie urządzeń, które \\
    przeprowadzając kolejne kroki \\
    \sout{zmieniają swój stan}
  \item logika, wyrażanie idei \\
    (definicje, twierdzenia, dowody, przykłady)
  \end{itemize}
\end{frame}


\begin{frame}{Analogia}
  \textbf{Do jakiej innej dziedziny ludzkiej działalności
    można porównać programowanie?}
  \begin{itemize}
  \item budowanie urządzeń\sout{, które \\
    przeprowadzając kolejne kroki} \\
    \sout{zmieniają swój stan}
  \item logika, wyrażanie idei \\
    (definicje, twierdzenia, dowody, przykłady)
  \end{itemize}
\end{frame}


\begin{frame}{Analogia}
  \textbf{Do jakiej innej dziedziny ludzkiej działalności
    można porównać programowanie?}
  \begin{itemize}
  \item \sout{budowanie urządzeń}\sout{, które \\
      przeprowadzając kolejne kroki} \\
    \sout{zmieniają swój stan}
  \item logika, wyrażanie idei \\
    (definicje, twierdzenia, dowody, przykłady)
  \end{itemize}
\end{frame}


\begin{frame}{Analogia}
  \textbf{Inne punkty wyjścia}
  \pause
  \begin{itemize}
  \item fizyka -- Joe Armstrong, Erlang
    \pause
  \item matematyka -- John Backus, FFP
    \pause
  \item językoznawstwo -- Noam Chomsky, gramatyka generatywna
  \end{itemize}
\end{frame}

\section{Przykład}

\begin{frame}{Przykład}

  \textbf{Program, który liczy sumę kwadratów
    początkowych siedmiu liczb pierwszych}

  \pause
  \texttt{licznik := 7\\*
liczba := 0\\*
suma := 0\\*
dopóki(licznik > 0):\\*
\ \ \ \ jeżeli jest\_pierwsza(liczba):\\*
\ \ \ \ \ \ \ \ suma := suma + liczba\^{}2\\*
\ \ \ \ \ \ \ \ licznik := licznik - 1\\*
\ \ \ \ liczba := liczba + 1
  }
   
\end{frame}

\begin{frame}{Perspektywa lingwistyczna}
  \textbf{suma kwardatów początkowych siedmiu liczb pierwszych} \\
  \pause
  ,,suma $X$''\pause, gdzie \\
  $X$ = ,,kwardaty $Y$''\pause, gdzie \\
  $Y$ = ,,$k$ początkowych liczb pierwszych''\pause, gdzie \\
  $k$ = 7
\end{frame}

\begin{frame}{Drzewko}

  \begin{center}
    \Tree [.suma
      [.kwadratów
        [.liczb-pierwszych siedmiu początkowych ] ] ]
  \end{center}
  
\end{frame}

\begin{frame}
  Do wyjaśnienia: \pause
  \begin{itemize}
  \item co znaczy ,,suma $X$''? \pause
  \item co znaczy ,,kwadraty $Y$'' \pause
  \item co znaczy ,,siedem początkowych liczb pierwszych''?
  \end{itemize}
\end{frame}

\begin{frame}{Znaczenie}
  \textbf{Co znaczy ,,siedem początkowych liczb pierwszych''?} \\
  \pause
  \begin{center}
    \begin{tabular}{ l c r }
      denotacja & vs. & konotacja \pause\\
    \end{tabular}
  \end{center}
  \begin{itemize}
  \item denotacja (odniesienie): obiekt, do którego dane wyrażenie językowe się
    \textit{odnosi}\pause, np. ciąg \texttt{(2 3 5 7 11 13 17)} \pause
  \item konotacja (określenie): inne wyrażenie językowe równoważne danemu
    (ale prostsze pojęciowo)
  \end{itemize}
  
\end{frame}

\begin{frame}{Redukcja}
  \textbf{Konotacja (definicja)}
 
  Niech $N$ i $M$ oznaczają liczby naturalne. Rozważmy znaczenie wyrażenia \\
  ,,$N$ najmniejszych liczb pierwszych większych od $M$'' \pause \\

  \begin{enumerate}
  \item wiemy, że znaczeniem tego wyrażenia (o ile jest sensowne)
    jest \textbf{ciąg} $N$-elementowy \pause
    
  \item znaczenie wyrażenia ,,$0$ najmniejszych liczb pierwszych
    większych od $M$'' jest \textbf{tożsame} (koekstensjonalne)
    ze znaczeniem wyrażenia ,,pusty ciąg''
  \end{enumerate}
    
\end{frame}

\begin{frame}{Redukcja}
  \textbf{Konotacja (definicja)} 
  Niech $N$ i $M$ oznaczają liczby naturalne. Rozważmy znaczenie wyrażenia \\
  ,,$N+1$ najmniejszych liczb pierwszych większych od $M+1$'' \pause \\

  \begin{enumerate}
    \setcounter{enumi}{2}
    \item jeżeli $M+1$ jest liczbą pierwszą, to znaczeniem wyrażenia
    ,,$N+1$ najmniejszych liczb pierwszych większych od $M$'' jest \textbf{ciąg},
    którego pierwszym elementem jest $M+1$, a którego pozostałe elementy
    to ,,$N$ najmniejszych liczb pierwszych większych od $M+1$ \pause
    
  \item jeżeli $M+1$ nie jest liczbą pierwszą, to znaczenie wyrażenia
    ,,$N+1$ najmniejszych liczb pierwszych większych od $M$'' jest
    \textbf{takie samo}, jak znaczenie wyrażenia ,,$N+1$ liczb pierwszych większych
    od $M+1$''.
  \end{enumerate}
\end{frame}

\begin{frame}
  \center{Uff...}
\end{frame}

\begin{frame}{Formalna notacja}

  \textbf{suma kwadratów początkowych siedmiu liczb pierwszych}
  \pause
  sum of squares of initial seven prime numbers 
  {\tiny (angielski nie ma deklinacji)} \\
  \pause
  \texttt{(sum (squares (initial seven prime numbers)))} {\tiny(złożone
    deskrypcje bierzemy w nawiasy, \texttt{f} of \texttt{x} = \texttt{(f x)})} \\
  \pause
  \textbf{\texttt{(sum (squares (prime-numbers seven initial)))}}
  {\tiny(słowa rządzące na początku)} \\

\end{frame}

\begin{frame}{Język Scheme}
  \texttt{
(define (prime-numbers amount lowest)\\*
\ \ (if (equal? amount 0)\\*
\ \ \ \ '()\\*
\ \ \ \ (if (prime? lowest)\\*
\ \ \ \ \ \ (cons lowest (prime-numbers (- amount 1)\\*
\ \ \ \ \ \ \ \ \ \ \ \ \ \ \ \ \
\ \ \ \ \ \ \ \ \ \ \ \ \ \ \ \ (+ lowest 1)))\\*
\ \ \ \ \ \ (prime-numbers amount (+ lowest 1)))))\\*
  }
\end{frame}

\begin{frame}{Język Scheme}
  \texttt{
(\textbf{define (prime-numbers amount lowest)}\\*
\ \ (if (equal? amount 0)\\*
\ \ \ \ '()\\*
\ \ \ \ (if (prime? lowest)\\*
\ \ \ \ \ \ (cons lowest (prime-numbers (- amount 1)\\*
\ \ \ \ \ \ \ \ \ \ \ \ \ \ \ \ \
\ \ \ \ \ \ \ \ \ \ \ \ \ \ \ \ (+ lowest 1)))\\*
\ \ \ \ \ \ (prime-numbers amount (+ lowest 1)))))\\*
  }
\end{frame}

\begin{frame}{Język Scheme}
  \texttt{
(define (prime-numbers amount lowest)\\*
\ \ \textbf{(if (equal? amount 0)\\*
\ \ \ \ '()\\*
\ \ \ \ (if (prime? lowest)\\*
\ \ \ \ \ \ (cons lowest (prime-numbers (- amount 1)\\*
\ \ \ \ \ \ \ \ \ \ \ \ \ \ \ \ \
\ \ \ \ \ \ \ \ \ \ \ \ \ \ \ \ (+ lowest 1)))\\*
\ \ \ \ \ \ (prime-numbers amount (+ lowest 1))))})\\*
  }
\end{frame}

\begin{frame}{Język Scheme}
  \texttt{
(define (prime-numbers amount lowest)\\*
\ \ (if \textbf{(equal? amount 0)\\*
\ \ \ \ '()}\\*
\ \ \ \ (if (prime? lowest)\\*
\ \ \ \ \ \ (cons lowest (prime-numbers (- amount 1)\\*
\ \ \ \ \ \ \ \ \ \ \ \ \ \ \ \ \
\ \ \ \ \ \ \ \ \ \ \ \ \ \ \ \ (+ lowest 1)))\\*
\ \ \ \ \ \ (prime-numbers amount (+ lowest 1)))))\\*
  }
\end{frame}


\begin{frame}{Język Scheme}
  \texttt{
(define (prime-numbers amount lowest)\\*
\ \ (if (equal? amount 0)\\*
\ \ \ \ '()\\*
\ \ \ \ \textbf{(if (prime? lowest)\\*
\ \ \ \ \ \ (cons lowest (prime-numbers (- amount 1)\\*
\ \ \ \ \ \ \ \ \ \ \ \ \ \ \ \ \
\ \ \ \ \ \ \ \ \ \ \ \ \ \ \ \ (+ lowest 1)))\\*
\ \ \ \ \ \ (prime-numbers amount (+ lowest 1)))}))\\*
  }
\end{frame}

\begin{frame}{Język Scheme}
  \texttt{
(define (prime-numbers amount lowest)\\*
\ \ (if (equal? amount 0)\\*
\ \ \ \ '()\\*
\ \ \ \ (if \textbf{(prime? lowest)\\*
\ \ \ \ \ \ (cons lowest (prime-numbers (- amount 1)\\*
\ \ \ \ \ \ \ \ \ \ \ \ \ \ \ \ \
\ \ \ \ \ \ \ \ \ \ \ \ \ \ \ \ (+ lowest 1)))}\\*
\ \ \ \ \ \ (prime-numbers amount (+ lowest 1)))))\\*
  }
\end{frame}

\begin{frame}{Język Scheme}
  \texttt{
(define (prime-numbers amount lowest)\\*
\ \ (if (equal? amount 0)\\*
\ \ \ \ '()\\*
\ \ \ \ (if (prime? lowest)\\*
\ \ \ \ \ \ (cons lowest (prime-numbers (- amount 1)\\*
\ \ \ \ \ \ \ \ \ \ \ \ \ \ \ \ \
\ \ \ \ \ \ \ \ \ \ \ \ \ \ \ \ (+ lowest 1)))\\*
\ \ \ \ \ \ \textbf{(prime-numbers amount (+ lowest 1))})))\\*
  }
\end{frame}

\begin{frame}{Pojęcia pierwotne}
  Użyte pojęcia pierwotne: \pause
  \begin{itemize}
  \item \texttt{(if <warunek> <wartość> <alternatywa>)} -- jeżeli
    \texttt{<warunek>} jest spełniony, znaczeniem całego wyrażenia jest
    \texttt{<wartość>}, a w przeciwnym razie jest nią \texttt{<alternatywa>}\pause
  \item \texttt{(equal? a b)} -- \texttt{a} i \texttt{b} są równe\pause
  \item \texttt{(+ a b)} -- suma \texttt{a} i \texttt{b}\pause
  \item \texttt{(- a b)} -- różnica \texttt{a} i \texttt{b}\pause
  \item \texttt{(cons element sequence)} -- sekwencja, której pierwszym
    elementem jest \texttt{element}, zaś pozostałe elementy to elementy
    sekwencji \texttt{sequence}\pause
  \item \texttt{'()} -- sekwencja pusta
  \end{itemize}
\end{frame}

\begin{frame}{Pojęcia wtórne}
  Użyte pojęcia wtórne:\pause
  \begin{itemize}
  \item \texttt{prime-numbers} -- ale to je właśnie definiujemy
    (rekurencyjnie) \pause
  \item \texttt{(prime? n)} -- test, czy \texttt{n} jest liczbą pierwszą
  \end{itemize}
\end{frame}

\begin{frame}{Pierwszość liczby}
  $n$ jest liczbą pierwszą, jeśli jej jedyne podzielniki to $1$ oraz $n$\pause
  \texttt{
(define (prime? n)\\*
\ \ (equal? (divisors n) (list 1 n)))\\*
    }\pause
  pojęcia pierwotne:\pause
  
  
  
\end{frame}


\end{document}
